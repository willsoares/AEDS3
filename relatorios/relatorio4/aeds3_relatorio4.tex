\documentclass[10pt,a4paper]{article}
\usepackage[utf8]{inputenc}
\usepackage[english]{babel}
\usepackage{amsmath}
\usepackage{amsfonts}
\usepackage{amssymb}
\usepackage{listings}
\usepackage{color}
\usepackage[left=2cm,right=2cm,top=2cm,bottom=2cm]{geometry}


\definecolor{dkgreen}{rgb}{0,0.6,0}
\definecolor{gray}{rgb}{0.5,0.5,0.5}
\definecolor{mauve}{rgb}{0.58,0,0.82}

\lstset{frame=tb,
  language=Java,
  aboveskip=3mm,
  belowskip=3mm,
  showstringspaces=false,
  columns=flexible,
  basicstyle={\small\ttfamily},
  numbers=none,
  numberstyle=\tiny\color{gray},
  commentstyle=\color{dkgreen},
  stringstyle=\color{mauve},
  breaklines=true,
  breakatwhitespace=true,
  tabsize=3
}

\author{Willian de Souza Soares \textendash\ 2014.1.08.044}
\title{Report 4 \textendash\ AEDs III}
\date{March 16, 2020}
\begin{document}
\maketitle
\pagenumbering{gobble}
\begin{abstract}
The Depth First Search, DFS, is an algorithm to identify if a value is allocated in the graph and, if it is, determine it's position. 
\end{abstract}
\section{The algorithm}
In short, the DFS can be described as an algorithm that looks for the deepest vertex $v$ in a graph and then, turning back to an parent vertex of $v$ and going for the deepest sibling of $v$, repeating this until all the vertices are visited or the search is completed.

For this algorithm, we can 'colour' the vertices one of three colors: 
\begin{list}{•}{}
\item White \textendash\ The vertex hasn't been visited yet;
\item Gray \textendash\ The vertex itself has been visited, but not all it's adjacents; 
\item Black \textendash\ The vertex and all it's adjacents has been visited.
\end{list}

We can markdown the time when the vertex was discovered (turned gray) and when it was closed (turned black).

\subsection{DFS \textendash\ Recursive implementation}
This implementation has a time complexity of $\theta(V)$ as $V$ the numbers of Vertices on the Graph.
\begin{lstlisting}
function DFS(Graph G){
	//All vertices must start on white
    foreach(Vertex u in Vertices(G))
        color[u] = WHITE
    time = 0;
    
    //Visiting all unvisited vertices of the graph G
    foreach(Vertex u in Vertices(G))
        if color[u] == WHITE
            VISIT(u)
}

function VISIT(Vertex u){
	//Marking this vertex as visited and its discovery time
    color[u] = GREY
    time += 1
    discoveryTime[u] = time
    
    //Visiting all unvisited adjacents of the vertex u
    foreach(vertex v in Adjacent(u))
        if color[v] == WHITE
            VISIT(v)
    
    //When the code reach here, the vertex u and all it's adjacents are visited.
    //Marking u as finished.
    color[u] = BLACK
    time += 1
    closeTime[u] = time
}
\end{lstlisting}

\subsection{DFS \textendash\ Iterative implementation}
This implementation has a time complexity of $\theta(E)$ as $E$ the numbers of Edges on the Graph.
\begin{lstlisting}
function DFS(Graph G, Vertex v){
    //All vertices must start on white
    foreach(Vertex u in G)
        color[u] = WHITE

    //Putting on the stack the first vertex to be visited
    time = 0
    stack.push(v)
    while(stack.hasNext())
        //Grab the first vertex on the stack.
        u = stack.pop()
        //If not visited
        if(color[u] == WHITE)
            //Mark as visited            
            time +=1
            color[u] = GREY
            discoveryTime[u] = time
            
            //Put all adjacents on stack to be visited
            foreach(Vertex v in adjacentEdges(u))
                S.push(v)

            //Mark u as closed
            time += 1
            color[u] = BLACK
            closeTime[u] = time
}
\end{lstlisting}


\end{document}
