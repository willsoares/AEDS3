\documentclass[10pt,a4paper]{article}
\usepackage[utf8]{inputenc}
\usepackage[english]{babel}
\usepackage{amsmath}
\usepackage{amsfonts}
\usepackage{amssymb}

\author{Willian de Souza Soares \textendash\ 2014.1.08.034}
\title{Report 3 \textendash\ AEDs III}
\begin{document}
\maketitle
\pagenumbering{gobble}
\begin{abstract}
Some of the computational representations that can be used on graphs, including \emph{Adjacency Matrix}, \emph{Incidence Matrix} and \emph{Adjacency List}.
\end{abstract}
\section{Adjacency Matrix}
Adjacency: $a$ is \textbf{adjacent} to $b$ if $a$ is connected to $b$.

An adjacency matrix is represented by a square matrix, having $n$ elements, $n \times n$ is the number of Vertices on a graph $G$.

$G=(V,E)$

$V = {a,b,c}$

$E = {(a,b), (a,c), (b,c), (c,a), (c,b)}$

$G$:
$
\begin{tabular}{l|lll}
           & \textbf{a} & \textbf{b} & \textbf{c} \\ \hline
\textbf{a} &            & 1          & 1          \\
\textbf{b} &            &            & 1          \\
\textbf{c} & 1          & 1          &           
\end{tabular}
$

\begin{list}{$ \bullet $}{Advantages:}
\item $\theta (1) $ for access;
\item Efficient when dealing with a complete graph.
\end{list}
\begin{list}{$ \bullet $}{Disadvantages:}
\item $\theta (V^{2}) $ for memory usage;
\item Can be under-used and be made mostly of empty spaces, when dealing with a sparse graph;
\item Cannot represent valued \textbf{and} parallel edges at the same time.
\end{list}
\newpage
\section{Incidence Matrix}
Given a graph $G=(V,E)$, the incidence matrix of $G$ has the size $|V| \times |E|$.

$G=(V,E)$

$V = {a,b,c}$

$E = {(a,b), (a,c), (b,c), (c,a), (c,b)}$

$G$:
$
\begin{tabular}{l|lllll}
           & \textbf{e0} & \textbf{e1} & \textbf{e2} & \textbf{e3} & \textbf{e4} \\ \hline
\textbf{a} & 1           & 1           &             & 1           &             \\
\textbf{b} & 1           &             & 1           &             & 1           \\
\textbf{c} &             & 1           & 1           & 1           & 1          
\end{tabular}
$

\begin{list}{$ \bullet $}{Advantages:}
\item $\theta (E) $ for access. Can be a disadvantage when dealing with too many edges;
\item Efficient when dealing with sparse graphs
\end{list}
\begin{list}{$ \bullet $}{Disadvantages:}
\item $\theta (V \times E) $ for memory usage: uses too many space when dealing with graphs with many Edges;
\end{list}


\section{Adjacency List}
Given a graph $G=(V,E)$, the graph $G$ is represented by an array $a_{v}$ of linked lists.
Each of these linked lists represents an edge $e$ adjacent to $v$.

$G=(V,E)$

$V = {a,b,c}$

$E = {(a,b), (a,c), (b,c), (c,a), (c,b)}$

$G$:
$
\begin{array}{ccccc}
a & \rightarrow & b & c & // \\ 
b & \rightarrow  & c & // &  \\ 
c & \rightarrow  & a & b & //
\end{array}
$

\begin{list}{$ \bullet $}{Advantages:}
\item Uses the lowest memory than the another options, $ \theta (V + E) $
\end{list}
\begin{list}{$ \bullet $}{Disadvantages:}
\item Slow for access and operations with edges, $\theta (E) $
\end{list}

\end{document}
